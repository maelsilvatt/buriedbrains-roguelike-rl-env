%%%%%%%%%%%%%%%%%%%%%%%%%%%%%%%%%%%%%%%%%%%%%%%%%%%%%%%%%%%%%%%%%%%%%%%%%%%%%%%%
% DOCUMENTO LATEX APRIMORADO
% Autor Original: Ismael S.
% Melhorias por: Gemini (Google AI)
% Data: \today
% Descrição: Este documento apresenta uma descrição técnica do problema de geração
%            de grafos com variabilidade controlada para ambientes de
%            Aprendizado por Reforço.
%%%%%%%%%%%%%%%%%%%%%%%%%%%%%%%%%%%%%%%%%%%%%%%%%%%%%%%%%%%%%%%%%%%%%%%%%%%%%%%%

\documentclass[a4paper, 11pt]{article}

% --- PACOTES ESSENCIAIS ---
\usepackage[utf8]{inputenc} % Codificação de entrada (padrão em distros modernas)
\usepackage[brazil]{babel}  % Suporte para o português do Brasil
\usepackage[T1]{fontenc}    % Codificação da fonte para melhor hifenização

% --- PACOTES DE LAYOUT E ESTILO ---
\usepackage{geometry}       % Para controle de margens
\geometry{
    a4paper,
    top=2.5cm,
    bottom=2.5cm,
    left=2.5cm,
    right=2.5cm
}
\usepackage{graphicx}       % Para inclusão de imagens (carregado pelo TikZ, mas é boa prática incluir)
\usepackage{subcaption}     % Para criar subfiguras (ex: a, b) dentro de um ambiente figure

% --- PACOTES DE GRÁFICOS E DIAGRAMAS ---
\usepackage{tikz}
\usetikzlibrary{
    arrows.meta,    % Para setas avançadas
    positioning,    % Para posicionamento relativo de nós (ex: right=of)
    backgrounds,    % Para desenhar no fundo (ex: caixas de ajuste)
    fit,            % Para criar nós que se ajustam a outros nós
    calc            % Para cálculos de coordenadas
}

% --- DEFINIÇÕES GLOBAIS DO TIKZ ---
% Esta seção centraliza todos os estilos para facilitar a manutenção e garantir consistência.
\tikzset{
    % Distância padrão entre os nós
    node distance=1.5cm and 1.2cm,
    % Estilo base para todos os nós (salas)
    base_sala/.style={
        rectangle,
        draw,
        thick,
        minimum width=2.6cm,
        minimum height=0.9cm,
        font=\sffamily\bfseries, % Fonte sem serifa e em negrito
        align=center
    },
    % Estilo para salas comuns
    sala/.style={
        base_sala,
        fill=gray!10,
    },
    % Estilo para nós que representam bifurcações (out-degree > 1)
    bifurcacao/.style={
        base_sala,
        fill=blue!15,
    },
    % Estilo para nós que representam afunilamentos (out-degree = 1)
    afunilamento/.style={
        base_sala,
        fill=orange!20,
    },
    % Estilo padrão para as arestas (caminhos)
    edge_style/.style={
        draw,
        -Stealth, % Tipo de seta mais moderno
        thick,
        shorten >=1pt, % Encurta a linha para não tocar no nó
        shorten <=1pt
    },
    % Estilo para anotações ou caixas de destaque
    anotacao/.style={
        draw,
        red,
        dashed,
        thick,
        font=\small\sffamily,
        inner sep=4pt
    }
}


% --- METADADOS DO DOCUMENTO ---
\title{\textbf{Controle de Variabilidade Estrutural em Grafos para Ambientes de Aprendizado por Reforço}}
\author{Ismael S.}
\date{\today}


\begin{document}

\maketitle

\begin{abstract}
\noindent
Este documento detalha o desafio de gerar ambientes procedurais baseados em grafos para agentes de Aprendizado por Reforço (RL) com memória. O objetivo central é desenvolver um gerador de mapas que possua \emph{variabilidade estrutural controlada}, evitando que os agentes explorem padrões previsíveis no design do ambiente, em vez de desenvolverem estratégias de aprendizado generalizáveis. Apresentamos a definição do problema, exemplos de cenários indesejados e o estado final almejado, que consiste em um sistema de geração de grafos com parâmetros ajustáveis para a frequência e distribuição de pontos de decisão (bifurcações) e corredores (afunilamentos).
\end{abstract}

\section{Contexto do Projeto}

O projeto consiste no desenvolvimento de um \textbf{ambiente de simulação} destinado a servir como \emph{benchmark} para agentes de Inteligência Artificial treinados via \textbf{Aprendizado por Reforço (RL)}, com foco em agentes dotados de memória.

\begin{itemize}
    \item Agentes de RL aprendem por meio de tentativa e erro, otimizando seu comportamento com base em um sinal de recompensa.
    \item Um \emph{benchmark} robusto deve avaliar a capacidade \textbf{genuína} de aprendizado e adaptação do agente, e não sua habilidade de explorar falhas ou padrões previsíveis no design do ambiente.
\end{itemize}

\section{O Ambiente: Mapas como Grafos Direcionados}

O ambiente é constituído por \textbf{mapas de salas} gerados proceduralmente a cada nova sessão de treinamento. Essa geração procedural garante que o agente não possa simplesmente memorizar um mapa estático. A estrutura do mapa é formalmente um \textbf{Grafo Acíclico Dirigido (DAG)}.

\begin{itemize}
    \item Cada \textbf{sala} é um \textbf{vértice} (nó) do grafo.
    \item Os \textbf{caminhos} entre as salas são as \textbf{arestas} (conexões) direcionadas.
    \item A progressão do agente é sempre unidirecional (sem ciclos), caracterizando uma estrutura \emph{top-down}.
\end{itemize}

Com base na estrutura de saídas de cada nó, definimos dois tipos de vértices cruciais para a topologia do grafo:

\begin{itemize}
    \item \textbf{Bifurcação:} Vértice com grau de saída (out-degree) maior que 1. Representa um ponto de decisão para o agente.
    \item \textbf{Afunilamento:} Vértice com grau de saída igual a 1. Representa um caminho único, sem escolha de rota.
\end{itemize}

\begin{figure}[h!]
    \centering
    \begin{tikzpicture}
        % Nós
        \node[sala] (s1) {Sala 1};
        \node[bifurcacao, right=of s1] (s2) {Sala 2};
        \node[afunilamento, below left=of s2] (s3a) {Sala 3A};
        \node[bifurcacao, below right=of s2] (s3b) {Sala 3B};
        \node[sala, below left=of s3b] (s4c) {Sala 4C};
        \node[afunilamento, below right=of s3b] (s4d) {Sala 4D};
        \node[sala, below=of s4d] (s5) {Sala 5};

        % Arestas
        \path[edge_style]
            (s1) edge (s2)
            (s2) edge (s3a)
            (s2) edge (s3b)
            (s3b) edge (s4c)
            (s3b) edge (s4d)
            (s4d) edge (s5);
    \end{tikzpicture}
    \caption{Exemplo da estrutura de um mapa, ilustrando \textbf{bifurcações} (nós azuis) e \textbf{afunilamentos} (nós laranjas).}
    \label{fig:exemplo_estrutura}
\end{figure}

\section{O Problema: Previsibilidade Estrutural}

Agentes de RL são proficientes em encontrar e explorar padrões. Se a geração procedural dos mapas resultar em grafos com uma \textbf{estrutura topológica previsível}, o agente pode desenvolver "atalhos" estatísticos, em vez de aprender uma política de decisão robusta.

Isso compromete a validade do \emph{benchmark}, pois o agente estaria explorando uma falha no gerador do ambiente, e não demonstrando inteligência generalizável.

\subsection{Cenário Problemático: Padrões Fixos}
Considere um gerador que, embora produza grafos diferentes, sempre posiciona os afunilamentos em níveis de profundidade específicos. O agente poderia aprender a explorar essa regularidade, por exemplo, antecipando a ausência de escolhas em determinados pontos da jornada.

\begin{figure}[h!]
    \centering
    \begin{tikzpicture}
        % Nós
        \node[bifurcacao] (s1) {S1};
        \node[sala, right=of s1] (s2) {S2};
        \node[afunilamento, below left=of s2] (s3a) {S3A};
        \node[sala, below right=of s2] (s3b) {S3B};
        \node[afunilamento, below=of s3b] (s4b) {S4B};
        \node[sala, below=of s3a] (s4a) {S4A};
        \node[afunilamento, below left=of s4a] (s5a) {S5A};
        \node[sala, below right=of s4a] (s5b) {S5B};

        % Arestas
        \path[edge_style]
            (s1) edge (s2)
            (s2) edge (s3a) (s2) edge (s3b)
            (s3a) edge (s4a)
            (s3b) edge (s4b)
            (s4a) edge (s5a) (s4a) edge (s5b);

        % Anotação do padrão previsível
        \node[anotacao, fit=(s3a)(s4b)(s5a), label={[red]below:Padrão de afunilamentos fixo/previsível}] {};
    \end{tikzpicture}
    \caption{Cenário indesejado onde a localização dos afunilamentos segue um padrão previsível, permitindo a exploração pelo agente.}
    \label{fig:cenario_problema}
\end{figure}

\section{Objetivo: Variabilidade Estrutural Controlada}

O objetivo é projetar um gerador de grafos que permita um \textbf{controle preciso sobre a variabilidade estrutural} do ambiente. A meta não é a aleatoriedade total, que poderia resultar em mapas insolúveis ou triviais, mas sim uma \textbf{imprevisibilidade gerenciável}.

Busca-se um sistema onde seja possível parametrizar a geração do grafo, ajustando, por exemplo:
\begin{itemize}
    \item A \textbf{probabilidade} de um novo nó ser uma bifurcação ou um afunilamento.
    \item A \textbf{distribuição} desses tipos de nós ao longo do grafo (ex: mais bifurcações no início e mais afunilamentos no final).
    \item A \textbf{dependência contextual}, onde a probabilidade de criar um afunilamento pode depender do tipo do nó anterior.
\end{itemize}

\subsection{Cenário Desejado: Gerações Diversificadas}
O gerador ideal deve ser capaz de produzir, a partir do mesmo conjunto de parâmetros probabilísticos, uma ampla gama de topologias de mapa. Isso força o agente a desenvolver estratégias que se adaptem à estrutura específica de cada mapa gerado, em vez de memorizar um padrão global.

\begin{figure}[h!]
    \centering
    % Subfigura (a)
    \begin{subfigure}[b]{0.48\textwidth}
        \centering
        \begin{tikzpicture}
            \node[bifurcacao] (s1) {S1};
            \node[sala, right=1cm of s1] (s2) {S2};
            \node[afunilamento, below left=1cm and 0.2cm of s2] (s3a) {S3A};
            \node[bifurcacao, below right=1cm and 0.2cm of s2] (s3b) {S3B};
            \node[sala, below left=1cm and 0.2cm of s3b] (s4c) {S4C};
            \node[afunilamento, below=1.5cm of s3a] (s4a) {S4A};
            \path[edge_style] (s1) edge (s2) (s2) edge (s3a) (s2) edge (s3b)
                              (s3a) edge (s4a) (s3b) edge (s4c);
        \end{tikzpicture}
        \caption{Geração 1: Menos bifurcações.}
        \label{fig:geracao1}
    \end{subfigure}
    \hfill % Espaço horizontal entre as subfiguras
    % Subfigura (b)
    \begin{subfigure}[b]{0.48\textwidth}
        \centering
        \begin{tikzpicture}
            \node[bifurcacao] (s1) {S1};
            \node[sala, right=1cm of s1] (s2) {S2};
            \node[bifurcacao, below left=1cm and 0.2cm of s2] (s3a) {S3A};
            \node[bifurcacao, below right=1cm and 0.2cm of s2] (s3b) {S3B};
            \node[afunilamento, below=1.5cm of s3a] (s4a) {S4A};
            \node[sala, below=1.5cm of s3b] (s4b) {S4B};
            \path[edge_style] (s1) edge (s2) (s2) edge (s3a) (s2) edge (s3b)
                              (s3a) edge (s4a) (s3b) edge (s4b);
        \end{tikzpicture}
        \caption{Geração 2: Mais bifurcações.}
        \label{fig:geracao2}
    \end{subfigure}
    \caption{Cenário desejado: O gerador produz mapas com topologias distintas, forçando o agente a se adaptar em vez de memorizar.}
    \label{fig:cenario_desejado}
\end{figure}

\section{Conclusão}

A questão central de pesquisa é, portanto, a formulação de um algoritmo de geração procedural de grafos que equilibre ordem e caos. O sistema deve prover um nível de \textbf{imprevisibilidade estrutural} suficiente para impedir a exploração de padrões, mas com \textbf{controle paramétrico} para garantir que os ambientes gerados permaneçam solucionáveis e propícios ao aprendizado. Acredito que uma abordagem baseada em probabilidades condicionais e distribuições ajustáveis seja um caminho promissor para alcançar este objetivo.

\vspace{1cm} % Espaço antes da assinatura

\noindent Atenciosamente,

\noindent Ismael S.

\end{document}