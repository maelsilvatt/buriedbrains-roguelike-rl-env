\documentclass[12pt, a4paper]{article}

\usepackage[utf8]{inputenc}
\usepackage[T1]{fontenc}
\usepackage{amsmath}
\usepackage{amssymb}
\usepackage{graphicx}
\usepackage[margin=2.5cm]{geometry}
\usepackage{lmodern}

\title{\textbf{Justificativa para a Modelagem Condicional da Função de Geração de Conteúdo Gamma}}
\author{Ismael S. Silva}
\date{8 de julho de 2025}

\begin{document}

\maketitle

\begin{abstract}
Este documento apresenta a justificativa e a arquitetura formal da função de geração procedural de conteúdo, denominada Gamma ($\Gamma$), para o ambiente de simulação multiagente \textit{BuriedBrains}. Argumentamos que, para testar rigorosamente hipóteses sobre comportamentos emergentes e sensibilidade ao risco, uma geração puramente aleatória é insuficiente, pois introduz ruído estocástico que obscurece os padrões de aprendizado. Em contrapartida, um modelo excessivamente determinístico falha em promover a generalização. Propomos uma arquitetura condicional que equilibra variabilidade e controle. A função $\Gamma$ utiliza um sistema de orçamento de dificuldade, sensível ao estado do jogo ($k$) e à topologia do grafo ($T$), para modular uma série de decisões probabilísticas. O resultado é a criação de uma "paisagem de risco e oportunidade" que é ao mesmo tempo dinâmica, coerente e rica em contexto, constituindo um componente fundamental do design experimental do ambiente.
\end{abstract}

\section{Introdução e Motivação}

O ambiente \textit{BuriedBrains} foi concebido como um laboratório computacional para o estudo de comportamentos sociais complexos em agentes de inteligência artificial. Um pilar central deste ambiente é a sua natureza procedural, que visa testar a capacidade de generalização e planejamento de longo prazo dos agentes. Para que este objetivo seja alcançado, a função de geração de conteúdo, $\Gamma$, não pode ser um mero mecanismo de aleatorização. Ela deve ser um componente integrante do design experimental.

A motivação para uma modelagem mais sofisticada surge da insuficiência de abordagens mais simples:
\begin{itemize}
    \item \textbf{Geração Puramente Aleatória:} Um modelo que seleciona conteúdo de forma arbitrária, mesmo que a partir de distribuições ponderadas, falha em criar uma curva de dificuldade coerente. O risco se torna uniforme e imprevisível, tornando impossível testar hipóteses sobre estratégias sensíveis ao risco (H2), pois o agente não tem um gradiente de risco para aprender.
    \item \textbf{Geração Estruturada Rígida:} Um modelo que sempre segue templates fixos, embora controlado, torna-se previsível e repetitivo. Isso compromete a necessidade de generalização (H4), pois o agente pode simplesmente memorizar os padrões, em vez de aprender as mecânicas subjacentes.
\end{itemize}

Diante disso, a função $\Gamma$ foi projetada para atuar como um "Mestre de Jogo" (Dungeon Master) algorítmico, que cria desafios contextuais em vez de obstáculos aleatórios.

\section{Arquitetura da Função Gamma}

A arquitetura escolhida para $\Gamma$ é sequencial, condicional e baseada em um sistema de economia interna (custo e orçamento). Ela decompõe a complexa tarefa de gerar uma sala em uma série de decisões probabilísticas menores e bem definidas.

\subsection{Estrutura Baseada em Slots e Categorias}

Para garantir controle estrutural, o vetor de atributos da sala, $\alpha(v)$, é definido como uma tupla de três elementos, onde cada elemento (ou "slot") corresponde a uma categoria de conteúdo:
$$\alpha(v) = (c_i, c_e, c_f)$$
Onde $c_i \in \text{Pool}_{\text{inimigos}} \cup \{\emptyset\}$, $c_e \in \text{Pool}_{\text{eventos}} \cup \{\emptyset\}$, e $c_f \in \text{Pool}_{\text{efeitos}} \cup \{\emptyset\}$. Esta estrutura impõe um limite máximo de um item por categoria, evitando a geração de salas caóticas e não balanceadas, e tornando o espaço de configurações de sala mais tratável.

\subsection{O Sistema de Orçamento e Custo}

O núcleo do mecanismo de controle é um sistema de economia de dificuldade.
\begin{itemize}
    \item \textbf{Custo:} A cada entidade de conteúdo $j$ é associado um $\text{Custo}(j) \in \mathbb{R}$, que quantifica seu impacto na dificuldade (positivo para desafios, negativo para recompensas).
    \item \textbf{Orçamento:} A cada sala $v$ é alocado um $\text{Orçamento}(v) \in \mathbb{R}^+$, que representa sua capacidade total de desafio.
\end{itemize}
Este sistema garante que desafios de alto custo (e.g., `Inimigo de Elite`) só possam aparecer em salas com um orçamento correspondentemente alto, estabelecendo uma relação fundamental entre a capacidade de uma sala e o conteúdo que ela pode abrigar.

\subsection{Condicionamento pela Topologia e Estado do Jogo}

A principal inovação do modelo é que tanto o orçamento quanto a probabilidade de preenchimento de cada slot são condicionados pelo estado do jogo e pela topologia do grafo. Seja $k$ o andar atual e $T = (\text{progresso}, \text{eh\_corredor})$ o vetor de informação topológica.

O orçamento inicial, $B_0$, é uma função determinística $B_0(k, T)$. Isso garante que salas em finais de andar (alto `progresso`) ou em posições estratégicas (não-corredores) recebam um orçamento maior, e vice-versa.

Adicionalmente, a probabilidade $p_{cat}$ de um slot de categoria ser preenchido também é uma função da topologia, $p_{cat}(T)$. Por exemplo, a probabilidade de um evento complexo ocorrer em um corredor é drasticamente reduzida, enquanto a chance de encontrar um inimigo em uma sala final é aumentada. Este condicionamento transforma a função $\Gamma$ em um gerador sensível ao contexto.

\section{Vantagens e Implicações para a Pesquisa}

A adoção desta arquitetura para a função $\Gamma$ oferece vantagens cruciais para os objetivos do \textit{BuriedBrains}.

\begin{itemize}
    \item \textbf{Criação de Paisagens de Risco Estruturadas:} Ao modular a dificuldade com base no progresso e na função da sala, o ambiente apresenta ao agente um gradiente de risco claro e aprendível. Isso é essencial para investigar se o agente desenvolve políticas prudentes e sensíveis ao risco.
    \item \textbf{Promoção de Generalização Sofisticada:} O ambiente exibe padrões contextuais consistentes (e.g., "corredores são mais seguros que salas finais"). Para ter sucesso, o agente deve aprender a reconhecer esses contextos e adaptar sua estratégia, em vez de apenas reagir a entidades individuais. Isso cria um teste robusto para a generalização de políticas.
    \item \textbf{Controle e Reprodutibilidade:} A natureza parametrizada do sistema (custos, pesos, fórmulas de orçamento) permite um balanceamento fino e a condução de experimentos controlados e reprodutíveis, que são a base de qualquer investigação científica rigorosa.
\end{itemize}

\section{Conclusão}
A modelagem da função $\Gamma$ como um processo condicional, estruturado e sensível ao contexto é uma decisão de design deliberada para alinhar o ambiente \textit{BuriedBrains} aos seus objetivos de pesquisa. Ela substitui o ruído de uma aleatoriedade pura pela complexidade rica de um sistema com coerência interna. Ao fazê-lo, transforma o ambiente em uma plataforma robusta, não apenas para jogar, mas para medir e analisar a emergência de estratégias inteligentes em face de um mundo dinâmico e cheio de nuances.

\end{document}